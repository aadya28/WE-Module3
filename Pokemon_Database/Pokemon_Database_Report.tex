\documentclass[10pt,onecolumn,letterpaper]{article}

%% Language and font encodings
\usepackage[spanish,english]{babel}
\usepackage[utf8x]{inputenc}
\usepackage[T1]{fontenc}

%% Sets page size and margins
\usepackage[a4paper,top=3cm,bottom=2cm,left=3cm,right=3cm,marginparwidth=1.75cm]{geometry}

%% Useful packages
\usepackage{amsmath}
\usepackage{graphicx}
\usepackage[colorinlistoftodos]{todonotes}
\usepackage[colorlinks=true, allcolors=blue]{hyperref}

%% Title
\title{
    %\vspace{-1in}     
    \usefont{OT1}{bch}{b}{n}
    \huge Designing a database for Pokemon!
}
\author{\large Aadya Srivastava \\ \\ GenAI Tool used: ChatGPT}

\begin{document}
\maketitle
\selectlanguage{english}

\section{Introduction}
This report documents the reflections/observations made while guiding GenAI through the process of designing a database for Pokémon, using both SQL (MySQL) and NoSQL (MongoDB) approaches.

\section{Problem Statement} 
The task involves designing a SQL and NoSQL database for Pokémon, comprising of creating tables, populating them with information provided such as Pokémon types, moves, and their attributes. Specific details include Pokémon types and their interactions, moves, and their powers. Additionally, queries need to be formulated to retrieve Pokémon capable of learning a certain move and moves effective against Grass-type Pokémon.

\section{Prompts given to GenAI}
\begin{itemize}
    \item Objective of the first chat: To give a structured prompt in fragments so as to not overwhelm GenAI. This was done by first giving some context on what we are trying to do, then moving on to designing a database using SQL by asking it to first create the tables, then insert data, then give SQL commands for the two queries. A similar approach was followed to get the commands for NoSQL database.  
    \item Objective of the second chat: To use a less structured approach to see if the GenAI's knowledge of Pokemon affects its responses.
    \item Objective of the third chat: To use a structured prompt but to ask GenAI to design NoSQL database first, to see if the sequence of designing databases impacts the responses. 
\end{itemize}

\section{Reflections on Responses Generated}
\begin{itemize}
    \item \textbf{Response to First Prompting Method:}
	In the initial approach, \textbf{GenAI provided accurate commands for all queries, effectively interacting with both SQL and NoSQL databases.} It demonstrated a clear understanding of database concepts and efficiently explained the differences between SQL and NoSQL technologies.
    
    \item \textbf{Response to Second Prompting Method:}
	In this approach, When opted a less structured approach and prompted to design the NoSQL database first, \textbf{the initial responses were somewhat vague.} Although, they did make sense conceptually, they didn't precisely align with what I had asked for. For instance, when prompted to create collections, instead of giving an executable command, GenAI provided a general outline of how the collection structure should look. Subsequently, when establishing the 'Pokemon' and 'Move' classic many-to-many relationship, \textbf{the method employed was not very efficient and led to redundant data storage}, although it ultimately achieved the desired outcome. \textbf{A significant challenge encountered was GenAI's prior knowledge of the Pokémon game, which influenced its responses} until I explicitly instructed it to only utilize the provided data.

     \item \textbf{Response to Third Prompting Method:}
     In this approach, \textbf{the main difference I observed was due to the inherent dissimilarity between SQL and NoSQL databases, particularly the absence of tables in NoSQL.}
     
     When prompted to “Create all the tables needed”, in case of SQL genAI provided an accurate query to create the necessary tables as per the prompt, however in case of NoSQL GenAI’s responses tended towards offering general strategies rather than specific commands for creating collections. Despite specific prompts, NoSQL responses remained vague until directly prompted for the command to create a collection. Once this clarification was given, GenAI successfully provided NoSQL commands, utilizing the recently received input data and prompts.
     
     In the case of SQL, although there was some improvement compared to the previous NoSQL responses, the focus on details was still lacking. Responses tended to provide a semblance of a solution rather than a precise command and when discrepancies were pointed out, GenAI started hallucinating and adding dummy data based on its prior Pokémon knowledge, requiring multiple attempts to generate the correct output.
\end{itemize}

\section*{Analysis and Conclusions}
Overall, GenAI does a great job when asked to stick to the provided data and its prior knowledge of the topic doesn’t interfere too much with its responses, otherwise it often starts hallucinating and adds random data to the prompts.

Also, it gives straightforward and precise responses when the prompts are clear, focus on only one task at a time, terminology used is accurate and when context is well described.

\section*{References}
\begin{itemize}
  \item ChatGPT Transcripts
  \begin{itemize}
    \item \url{https://chat.openai.com/share/3c4ce08c-b58b-4cc8-a356-41968320775a}
    \item \url{https://chat.openai.com/share/1e5e6a61-7e40-4c9f-90d2-bcaa73ac579a}
    \item \url{https://chat.openai.com/share/abe50515-c252-4bfe-b910-170205a254b0}
  \end{itemize}
  
  \item Google Drive Result Link
   \begin{itemize}
     \item SQL Query Result: \url {https://drive.google.com/file/d/14Gec9EfcD0dwrAqQ632Y2Kw5WVfwn3-g/view?usp=sharing}
     \item NoSQL Query Result: \url {https://drive.google.com/file/d/1pDE86aqxLblvO4W7hatpvGqvZZjFtf2c/view?usp=sharing}
  \end{itemize}
\end{itemize}

\end{document}